\section{Conclusión}

El objetivo principal de este proyecto fue el desarrollo del código de
integración numérica de las ecuaciones del equilibrio hidrostático en el Sol. A
pesar de que este fue cumplido con el código en \verb|EquilibrioHidrostatico|
una meta secundaria era experimentar con la programación de GPUs. Este paradigma
permite acceso a la tarjeta de video de una computadora, la cual está optimizada
para hacer miles de cálculos simultáneos. Aunque es poco probable que este
programa vea una mejora apreciable en su velocidad la programación en GPUs es
una de las áreas poco exploradas dentro de las implementaciones actuales de
código de simulación científicas. Esto sería una meta para el código en un
subsecuente trabajo.

En cuanto a las gráficas producidas en este trabajo se puede apreciar los
perfiles del plasma solar en la fotosfera, incluyendo la cromosfera y una
pequeña región de la corona. En estas uno de los detalles más interesantes es el
aumento de temperatura repentino en la región de transición a la corona, en la
cual también se observa un decaimiento rápido de la presión y densidad. Este
fenómeno ha sido documentado a lo largo de los años, sin una respuesta
definitiva. \cite{coronaHeatingProblem} La cuestión del motor de este
comportamiento cae fuera de este trabajo. Sin embargo se puede observar estas
regiones de calentamiento repentino en ambos modelos, incluyendo su efecto en la
densidad y presión del plasma.